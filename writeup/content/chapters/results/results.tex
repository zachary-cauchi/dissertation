\chapter{Results}
\label{chp:results}

In this chapter, the results of the experiments described in Section~\ref{sec:experiments} will be presented.
A discussion of the results and the challenges encountered will also be presented.

\section{Evaluation results}
\label{sec:evaluation-results}

\begin{table}[]
\begin{tabularx}{\linewidth}{rccccccc}
    \toprule
    \multicolumn{8}{c}{Experiment results} \\ \midrule
     & Q\rightarrow{}A & Q\rightarrow{}A & QA\rightarrow{}R & QA\rightarrow{}R & QAp\rightarrow{}R & Q\rightarrow{}AR & Q\rightarrow{}AR \\
    Wrd Embed. & Shr. & Sep. & Shr. & Sep. & Shr. & Shr & Sep. \\
    Rand Guess & 25\% & 25\% & 25\% & 25\% & 25\% & 6.25\% & 6.25\% \\
    Glove-300 & 52\% & 51\% & 25\% & 47\% & 26\% & 6.4\% & 6.4\% \\
    BERT-768 & 63\% & 63\% & 60\% & 60\% & 60\% & 24.7\% & 24\% \\
    w2v-300 & 33\% & 34\% & 25\% & 32\% & - & - & - \\
    w2v-768 & 32\% & 34\% & 25\% & 32\% & - & - & - \\
\end{tabularx}
\captionsource(Evaluation results)
    {Evaluation results from running the model across combinations of different token embeddings, VCR task types, and layout generator configurations. \textit{Shr.} refers to models with a shared \gls{bilstm} while \textit{Sep.} refers to one \gls{bilstm} per input sentence. \label{tab:my-table}}
    {Original performance results obtained for this dissertation.}
\end{table}
\todo[inline]{Include table notes from logbook in discussion.}

\todo[inline]{Include ablations section, discussing the ablations made in more detail (testing non-contextual embeddings such as GLOVE, smaller batches from own limitations, using a single LSTM versus multiple LSTMs, etc.)}

\section{Challenges}
\label{sec:experiment-challenges}

Several errors were encountered during training.
The most common error encountered was unstable learning which resulted in NaN loss errors or slow learning.
NaN losses were frequent when training the model using word2vec embeddings and didn't occur at all when using glove or bert.
Slow learning rates were observed during glove and word2vec training, especially with Q\rightarrow{}AR training.
Another factor in getting good predictions was batch size, which was strained by the task size and amount of data involved.
To work around the limited batch size available, training on QA\rightarrow{}R and Q\rightarrow{}AR modes was performed on a multi-gpu setup to allow for increased batch size.
This setup produced better learning rates and prediction performance compared to single-gpu training, at the cost of increased runtime due to the synchronisation overhead of keeping the two gpu models in mirrored.
