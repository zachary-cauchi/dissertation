\graphicspath{{content/chapters/2_background/figures/}}

\chapter{Background}%
\label{chp:background}

Figure~\ref{fig:sample} shows a sample figure and how to cross-reference
figures.
%
\begin{figure}[htbp]
    \centering
    \includegraphics[width=\textwidth,keepaspectratio]{sample}
    \caption[Short sample caption.]{Longer caption that  and shows below the figure.\label{fig:sample}}
\end{figure}
%
Lorem ipsum dolor sit amet, consectetur adipiscing elit.
Mauris sed ipsum risus.
Nulla aliquet quis quam sed eleifend.
Donec rutrum, dolor id vulputate pharetra, nulla tortor laoreet nisl,
pellentesque dapibus velit dolor suscipit purus.
Phasellus vitae eleifend sem.
Integer ultricies ex in neque pellentesque, vitae facilisis orci aliquam.
In pellentesque mollis turpis, eu tristique lacus eleifend nec.
Vestibulum orci neque, rhoncus vitae convallis eu, suscipit quis dui.
Nulla libero elit, porta sit amet sagittis vel, placerat sit amet tortor.
Aliquam hendrerit dolor sit amet sollicitudin ornare.
Aliquam placerat sodales est, in vestibulum nisl efficitur in.
Nulla venenatis aliquam sem, at volutpat nisl pellentesque eleifend.
Praesent vitae euismod nulla, eget vehicula turpis.
Duis quis tellus vitae nisi tempus tincidunt.

\section{Literature Review}%
\label{sec:literature_review}

Vivamus sit amet orci erat.
Morbi eleifend velit purus, sed gravida metus ullamcorper sed.
Aliquam sit amet interdum nulla, in aliquam diam.
Aliquam non libero tortor.
Nulla imperdiet dolor vel justo semper, ut efficitur enim varius.
Donec ultrices odio id orci fringilla tristique.
Ut fringilla nec felis a finibus.
Sed a felis sed odio elementum porta a ac nisl.
Curabitur suscipit, sem et facilisis tempus, nisl elit vestibulum eros, in
varius dolor enim vitae ante.
Vestibulum ante ipsum primis in faucibus orci luctus et ultrices posuere cubilia
curae; Nullam condimentum tempor consectetur.
Aliquam non porta nisi.
Proin molestie tincidunt tellus, id varius nibh finibus eget.

Vestibulum et neque erat.
Curabitur metus velit, dictum non vehicula vitae, sodales sed purus.
In mattis a mauris nec imperdiet.
Duis volutpat mi eget egestas placerat.
Vivamus non purus erat.
Cras quis egestas libero.
Sed id diam at enim vehicula porttitor.

Mauris tincidunt elementum porttitor.
Curabitur eu elit et metus luctus ultrices.
Aenean varius orci in turpis consectetur efficitur.
Quisque lacinia sagittis pharetra.
Aliquam efficitur aliquam arcu, vel ullamcorper tortor volutpat nec.
Curabitur sit amet semper tortor.
Vestibulum ante ipsum primis in faucibus orci luctus et ultrices posuere cubilia
curae; Cras at leo aliquet, porta mauris nec, pharetra augue.
Nam volutpat eu urna in ullamcorper.
Aliquam ultrices condimentum odio id eleifend.
Mauris tellus felis, mattis et pellentesque ac, laoreet vitae eros.
Aliquam at nisl lorem.
Quisque consequat ligula nec tellus ornare eleifend.

\section{Instructions}%
\label{sec:instructions}

This sentence refers to Section~\ref{sec:literature_review}, as an example of
how to do cross-referencing.
Equation~\eqref{eq:emc} shows one of the most famous equations.
%
\begin{equation}
    \label{eq:emc}
    e = mc^2
\end{equation}
%
An example of how to create multiple equations, where they all align is given
below.
%
\begin{align}
    e & = mc^2          \\
    m & = \frac{e}{c^2}
\end{align}

\section{Inserting references}%
\label{sec:inserting_references}

To insert a reference, the entry must be inserted in the \texttt{references.bib}
file.
The key or unique ID of the entry is then used to refer to it.
\LaTeX{} will automatically number the entry and generate the list of
references.
The paper in~\cite{sample_key} is used as a referencing example.

\section{Inserting acronyms}%
\label{sec:inserting_acronyms_and_glossary_entries}

The \gls{tcp} and \gls{udp} protocols are two layer 4 protocols, used for
demonstrating how to use acronyms.
On the second use of an acronym, only its initials are shown as demonstrated in
the following sentence.
The \gls{tcp} and \gls{udp} protocols are two layer 4 protocols, used for
demonstrating how to use acronyms.

\section{Using glossary terms}%
\label{sec:using_glossary_terms}

Let \gls{G} represent a loop-free directed graph, where \gls{V} and \gls{E}
represent the set of nodes and edges, respectively.

\section{Inserting a table}%
\label{sec:inserting_a_table}

A simple table is shown in Table~\ref{tab:simple}, with a more complex example
given in Table~\ref{tab:complex}.

\begin{table}
    \caption{Simple table example.\label{tab:simple}}
    \centering
    \begin{tblr}{|c|S[table-format=3.2]|c|}
        \hline
        \textbf{Header 1} & \textbf{Header 2} & \textbf{Header 3} \\
        \hline
        1                 & 2.3               & Orange            \\
        2                 & 100.5             & Blue              \\
        3                 & 35.0              & Black             \\
        \hline
    \end{tblr}
\end{table}

\begin{table}
    \centering
    \caption{Complex table example.\label{tab:complex}}
    \begin{tblr}{|Q[m,0.2\textwidth]|Q[m,0.2\textwidth]|Q[m,0.2\textwidth]|Q[m,0.2\textwidth]|}
        \hline
        \SetCell[r=2]{c} Table Head & \SetCell[c=3]{c} Table Column Head & & \\
        \hline
        & Table column subhead 1 & Table column subhead 2 & Table column subhead 3 \\
        \hline
        Item 1 & 2 & 3 & 4 \\
        \hline
        Item 2 & 2 & 3 & 4 \\
        \hline
        Item 3 & 2 & 3 & 4 \\
        \hline
        Item 4 & 2 & 3 & 4 \\
        \hline
    \end{tblr}
\end{table}

\section{Inserting code snippet}%
\label{sec:inserting_code_snippet}

The code snippet in Listing~\ref{lst:python_example} demonstrates a very simple
python program.

\begin{lstlisting}[language=Python,caption=Python example,label={lst:python_example}]
def main() -> None:
    print("Hello World")

if __name__ == "__main__":
    main()
\end{lstlisting}

\section{Inserting theorems, corollaries and lemmas}%
\label{sec:inserting_theorems,_corollaries_and_lemmas}

\begin{theorem}
    Let \(f\) be a function whose derivative exists in every point, then \(f\) is
    a continuous function.
\end{theorem}

\begin{theorem}[Pythagorean theorem]
    \label{pythagorean}
    This is a theorem about right triangles and can be summarised in the next
    equation
    \[ x^2 + y^2 = z^2 \]
\end{theorem}

And a consequence of theorem \ref{pythagorean} is the statement in the next
corollary.

\begin{corollary}
    There's no right rectangle whose sides measure 3cm, 4cm, and 6cm.
\end{corollary}

You can reference theorems such as \ref{pythagorean} when a label is assigned.

\begin{lemma}
    Given two line segments whose lengths are \(a\) and \(b\) respectively there is a
    real number \(r\) such that \(b=ra\).
\end{lemma}

\section{Inserting an algorithm}%
\label{sec:inserting_an_algorithm}

The pseudocode for a basic \gls{ea} using NSGA-II is given in
Algorithm~\ref{alg:evolutionaryAlgorithm}.

\begin{algorithm}
    \caption{Pseudocode for an Evolutionary Algorithm}%
    \label{alg:evolutionaryAlgorithm}
    \begin{algorithmic}
        \State \( \mathcal{P} \) = Population Size
        \State \( \chi \) = Number of Generations
        \State \( \omega \) = Crossover Probability
        \State \( \psi \) = Mutation Probability
        \State
        \State population = GenerateInitialPopulation(\( \mathcal{P} \))
        \For{\( 1, 2, \ldots, \chi \)}
        \State offspring = TournamentSelection(population, \( \mathcal{P} \))
        % Crossover
        \For{\(c_i \in \) offspring, \( i = 1, 3, 5, \ldots, \mathcal{P}\)}
        \State \(z\) = random(0, 1)
        \If{\( z < \omega \)}
        \State Crossover(\( c_i, c_{i+1} \))
        \EndIf
        \EndFor
        % Mutation
        \For{\(c_i \in \) offspring}
        \State \(z\) = random(0, 1)
        \If{\( z < \psi \)}
        \State Mutate(\( c_i \))
        \EndIf
        \EndFor
        % Calculate fitness
        \State CalculatePopulationFitness(offspring)
        % Update population size
        \State population = NSGA-II([population + offspring], \(\mathcal{P}\))
        \EndFor
    \end{algorithmic}
\end{algorithm}
