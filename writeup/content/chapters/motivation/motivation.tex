\chapter{Introduction}
\label{chp:introduction}

The \acrlong{vqa} problem --- which is a computer vision-language task whereby a system, given a question in the presence of an image, can predict an answer to the question \cite{agrawal_vqa_2016} --- has been leading up to a new problem: \acrfull{vcr}.
The \acrshort{vcr} problem extends the \acrshort{vqa} problem through the complexity of the questions being asked, which require more knowledge and insight to answer than is otherwise immediately apparent in a given image \cite{zellers_recognition_2019}.
Datasets are available for both tasks, and there are numerous \gls{ml} models which have been trained for both tasks.

A class of \gls{ml} models targetting \gls{vqa} tasks known as `compositional models'\cite{andreas_neural_2016} have proven to perform well on \gls{vqa} datasets\cite{fishandi_neural_2023}.
Such performance is attributed to the nature of their design whereby multiple smaller \gls{ml} modules are used to divide and conquer the steps for solving a \gls{vqa} task.
To further explore the use of compositional models in such tasks, we will be looking towards taking an existing model and adapting it to solve tasks that require \gls{vcr}.

While any compositional model could have been chosen for this work, the below characteristics were established to one model:

\begin{itemize}\label{list:reasons_for_nmn}
    \item The source code for the model and its distribution are available by the original authors along with steps for reproducing their results.
    \item The architecture of the model is such that each step taken to solve a \gls{vqa} task is performed in a sequential manner which can be viewed at each individual step and should therefore be easier to interpret when compared to other non-compositional models.
          This same behaviour can also be ported to \gls{vcr} tasks which should allow for better exploration of model performance on the task.
    \item The modular nature of the model architecture means future work can expand on its ability to solve \gls{vcr} tasks without necessitating a complete redesign to the model architecture.
    \item The chosen model is fully differentiable, meaning it can be trained without reinforcement learning or supervision of any kind (such as expert layouts) and produce comparable performance to models trained with layout supervision.
\end{itemize}

With the above in mind, the below objectives were established:

\begin{itemize}\label{list:list_of_objectives}
    \item Obtain a working copy of the of the model.
    \item Confirm the model operates as intended by training it on \gls{vqa} and produce accuracy results matching those published by the model authors (within a reasonable margin).
    \item Modify the model to be able to train and evaluate on the \gls{vcr} dataset.
    \item Perform experiments on the model to test whether certain modifications will produce better results or not.
    \item Following an analysis of its performance, outline future work that may expand upon the findings.
\end{itemize}

This study will begin with a review of the literature available, covering the datasets available for these task types, the models which target these datasets, and a discussion of which model best meets the above criteria for use in this study.
Following the literature review will be the methodology of how this model is set up to run on the \gls{vcr} dataset, including what experiments will be run.
The results of these experiments will be explored along with a qualitative analysis of how the model compared to other \gls{vcr} models covered in the literature review.
To conclude, a retrospective and discussion of future work will be presented.
